%----------------------------------------------------------------------------------------
%	PACKAGES AND OTHER DOCUMENT CONFIGURATIONS
%----------------------------------------------------------------------------------------

\documentclass[a4paper, 12pt]{article}
\usepackage[a4paper, left=30mm]{geometry}
\geometry{
	bmargin=2cm,
	tmargin=2cm,
	lmargin=3cm,
	rmargin=2cm,
}
\usepackage[utf8]{vietnam}
\usepackage{graphicx}
\usepackage{lastpage}

\usepackage{fancyhdr}
\pagestyle{fancy}
\fancyhf{}
\lhead{Phân tích cảm xúc trong văn bản y khoa}
\renewcommand{\sectionmark}[1]{\markboth{#1}{}}
\fancyhead[L]{\leftmark}
\rhead{\small \textit{Phân tích cảm xúc trong văn bản y khoa} \hspace{0.6cm} \thepage}
\setlength{\headheight}{30pt} 
\renewcommand{\headrulewidth}{0.0pt}
\renewcommand{\footrulewidth}{0.0pt}

\usepackage{subcaption}
\usepackage{color}
\usepackage{float}
\usepackage[labelfont=bf]{caption}
\usepackage{array}
\setlength\extrarowheight{2pt}
\newcolumntype{P}[1]{>{\centering\arraybackslash}p{#1}}
\usepackage{csquotes}
\usepackage[backend=bibtex,
style=numeric,
bibencoding=ascii
%style=alphabetic
%style=reading
]{biblatex}

\usepackage{amsthm}
%For noitemsep
\usepackage{enumitem}
%For tabular env
\usepackage{array}
%For footnote on table
\usepackage{tablefootnote}
%Reset footnote counter for each page
\usepackage[perpage]{footmisc}
\usepackage{changepage}   % for the adjustwidth environment
\usepackage{xspace}	%Stop ignoring space after command
\usepackage{import} %For section from another file
\usepackage[intlimits]{amsmath}
\usepackage{multirow}
\usepackage{color}
\usepackage{listings}
\definecolor{backcolour}{rgb}{0.95,0.95,0.95}
\definecolor{textcolour}{rgb}{0,0,0}
\lstset {
language=Python,
backgroundcolor=\color{backcolour},
numbers=right,
stepnumber=1,
showstringspaces=false,
    tabsize=1,
    numberstyle=\tiny\color{textcolour},
    breaklines=true,
    breakatwhitespace=false,
}

\usepackage{chngcntr}
\usepackage{amsmath}
\newcommand*{\Comb}[2]{{}^{#1}C_{#2}}% 
 
\counterwithin{figure}{section}
\counterwithin{table}{section}

\renewcommand{\ttdefault}{pcr}
\newcommand{\myquote}[1]{{\ttfamily #1}}

\renewcommand{\headrulewidth}{1pt}
\setlength\parindent{0pt} % noindent for all paragraph
\addbibresource{reference/my_ref.bib}

\captionsetup[figure]{labelfont=sc, font={sf}}
\captionsetup[table]{labelfont=sc, font=sf}

\theoremstyle{definition}
\newtheorem{exmp}{Ví dụ}[section]
\renewcommand{\thefootnote}{\arabic{footnote}}
\newcommand{\refformat}[1]{#1}
\newcounter{example}[section]

\newcommand{\example}[2]{
	\begin{minipage}{\textwidth}
	\vspace{6px}
	Ví dụ [section]
	
	\begin{adjustwidth}{0.08\textwidth}{0cm}
	\begin{minipage}{0.86\textwidth}
	 #2
	\end{minipage}
	 
	\end{adjustwidth}\vspace{10px}
	\end{minipage}
	}
%\newcommand{\myquote}[1]{#1}
% Constant
\newcommand{\tichcuc}{\textit{tích cực}\xspace}
\newcommand{\tieucuc}{\textit{tiêu cực}\xspace}
\newcommand{\trungtinh}{\textit{trung tính}\xspace}
\newcommand{\term}[1]{\textit{#1}}
\newcommand{\code}[1]{\texttt{#1}}
\begin{document}

\begin{titlepage}

\newcommand{\HRule}{\rule{\linewidth}{0.5mm}} % Defines a new command for the horizontal lines, change thickness here

\center % Center everything on the page
 
%----------------------------------------------------------------------------------------
%	HEADING SECTIONS
%----------------------------------------------------------------------------------------

\textsc{\large ĐẠI HỌC QUỐC GIA TP. HỒ CHÍ MINH}\\ [0.2cm]
\textsc{\large TRƯỜNG ĐẠI HỌC BÁCH KHOA}\\[0.3cm] % Name of your university/college
\textsc{\Large \scshape khoa khoa học và kỹ thuật máy tính}\\[0.5cm] % Major heading such as course name
\begin{figure}[H] 
\centering
\includegraphics[scale=0.4]{hinh/HCMUT_official_logo.png}
\end{figure} 

\textsc{\large LUẬN VĂN TỐT NGHIỆP ĐẠI HỌC}\\[0.2cm] % Minor heading such as course title

%----------------------------------------------------------------------------------------
%	TITLE SECTION
%----------------------------------------------------------------------------------------

\HRule \\[0.4cm]
{ \huge \bfseries PHÂN TÍCH CẢM XÚC\\TRONG VĂN BẢN Y KHOA}\\[0.2cm] % Title of your document
\HRule \\[0.8cm]

%----------------------------------------------------------------------------------------
%	AUTHOR SECTION
%----------------------------------------------------------------------------------------
\begin{flushright}
\begin{minipage}{0.7\textwidth}

\end{minipage}
\end{flushright}

\begin{flushleft} \large
\textbf{HỘI ĐỒNG: KHOA HỌC MÁY TÍNH}\\[1.0cm]
\end{flushleft}

\begin{flushleft} \large
\textbf{Giáo viên hướng dẫn:}\\
GS. TS. Cao Hoàng Trụ\\[1.0cm]
\end{flushleft}

\begin{flushleft} \large
\textbf{Giáo viên phản biện:}\\
GS. TS. Phan Thị Tươi\\[1.0cm]
\end{flushleft}

\begin{flushleft} \large
\textbf{Sinh viên thực hiện:}\\
Nguyễn Đức Trí (51204052)\\
Nguyễn Diệp Phương Linh (51201899)\\[1.5cm]
\end{flushleft}

\large \emph{Thành phố Hồ Chí Minh, 12/2016}

\vfill % Fill the rest of the page with whitespace
\end{titlepage}

\pagebreak
\section*{Lời cam đoan}
\pagebreak
\section*{Lời cảm ơn}
\pagebreak
\section*{Tóm tắt luận văn}
Nhận biết sự phân cực của các báo cáo y khoa rất quan trọng trong việc sàng lọc, tiếp thu và tổng hợp tri thức y học của y bác sĩ trước khi ra quyết định lâm sàng. Chúng tôi đã xem xét vấn đề này như một bài toán phân loại với dữ liệu đầu vào là một câu trong báo cáo y khoa và đầu ra là kết quả phân cực của câu: \tichcuc, \tieucuc hoặc \trungtinh. Để giải quyết bài toán này, chúng tôi kết hợp các kỹ thuật xử lý ngôn ngữ tự nhiên và học máy để xây dựng bộ phân loại cảm xúc trên câu. Dựa trên kết quả của \cite{niu2005analysis}, chúng tôi tiến hành rút trích 3 đặc trưng gồm có N-GRAM (kết hợp cả unigram và bigram), CHANGE PHRASE và đặc trưng phủ định để phục vụ xây dựng hệ thống phân tích tính phân cực cảm xúc. Đồng thời chúng tôi đề xuất kết hợp thêm một đặc trưng mở rộng SO-CAL vào hệ thống. Kết quả thí nghiệm cho thấy đặc trưng SO-CAL giúp cải thiện hiệu quả phân loại. Hệ thống phân loại chúng tôi xây dựng đạt độ chính xác 70.7\% trên tập dữ liệu 600 câu. Từ kết quả đạt được, chúng tôi hy vọng đề tài sẽ là cung cấp nhiều thông tin hữu ích, làm nền tảng cho các nghiên cứu sau này.
% START MAIN DOCUMENT

\pagebreak
\tableofcontents

\thispagestyle{empty}
\pagebreak
\listoffigures
\pagebreak
\listoftables

\pagebreak
\import{sections/}{gioi-thieu.tex}
\pagebreak

\import{sections/}{cac-cong-trinh-lien-quan.tex}
\pagebreak

\import{sections/}{kien-thuc-nen-tang.tex}
\pagebreak

\import{sections/}{phuong-phap-de-xuat.tex}
\pagebreak

\import{sections/}{hien-thuc-he-thong.tex}
\pagebreak

\import{sections/}{thi-nghiem-va-phan-tich.tex}
\pagebreak

\import{sections/}{ket-luan.tex}
\pagebreak

\printbibliography[title={Tài liệu tham khảo}]
\pagebreak

\section*{PHỤ LỤC (nếu có)}
\end{document}
