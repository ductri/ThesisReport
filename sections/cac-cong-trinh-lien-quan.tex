\section{Các công trình liên quan}
Đề tài phân tích cảm xúc (hay khai phá ý kiến) đã phát triển từ đầu thế kỷ 21. Tính ứng dụng cao của chủ đề này khiến cho nó nhanh chóng trở thành chủ đề hấp dẫn trong lĩnh vực xử lý ngôn ngữ tự nhiên. Chủ đề phân tích cảm xúc được phân nhỏ thành nhiều bài toán con như: phân loại tính phân cực (\tichcuc hay \tieucuc), phân loại ý kiến chủ quan với ý kiến khách quan, phân tích biểu cảm (vui, buồn, giận hờn, \ldots), \ldots\\

Có nhiều phương pháp đã được áp dụng, được chia thành 3 nhóm phương pháp\cite{Silva2015}: phương pháp dựa trên học máy, phương pháp dựa trên từ vựng và phương pháp sử dụng kết hợp cả 2 (Hình \ref{fig:cac-pp-phan-tich}). Bài toán phân tích cảm xúc đã được nghiên cứu trên rất nhiều lĩnh vực khác nhau: nhận xét, bình luận phim, bình luận trên các trang blog, các bình luận về sản phẩm, bình luận trên các diễn đàn về sức khỏe, các tweet trên mạng xã hội tweeter, các văn bản y khoa, \ldots
%Các phương pháp được sử dụng cũng tương tự, tuy nhiên với các nguồn dữ liệu khác nhau, bài toán con này có những cách định nghĩa khác nhau về cảm xúc, từ đó dẫn đến việc sử dụng các phương pháp và kết quả cũng khác nhau.
\begin{figure}
\centering
\includegraphics[scale=0.25]{../hinh/cac_pp_phan_tich.png}
\caption{Các phương pháp phân tích cảm xúc trong ngữ cảnh chung}
\label{fig:cac-pp-phan-tich}
\end{figure}
\subsection{Phương pháp dựa trên học máy}
Hướng tiếp cận dựa trên phương pháp học máy có thể áp dụng bất kỳ giải thuật học máy nào: Naive Bayse, SVM, Maximun Entropy, \ldots, trong đó nổi bật nhất là phương pháp Naive Bayse và SVM. \cite{pang2002thumbs} thử nghiệm cả 3 giải thuật Naive Bayse, Maximum Entropy và SVM, sử dụng các đặc trưng uni-gram, bi-gram và pos tagging để giải quyết bài toán phân loại nhị phân: phân loại các bình luận, nhận xét về phim có tính tính cực hay tiêu cực. Nghiên cứu đó đạt kết quả tốt nhất là 81.6\% với giải thuật SVM. Không giống như \cite{pang2002thumbs}, Pak và Paroubek giải quyết bài toán phân loại đa lớp. Nhóm tác giả phân tích các tweeter và phân loại chúng là tích cực, tiêu cực hay trung tính. Nghiên cứu cùng thử nghiệm SVM, MNB, CRF sử dụng các đặc trưng uni-gram, bi-gram, n-gram, và vị trí các n-gram. Kết quả tốt nhất nhóm tác giả đạt được khi kết hợp MNB với đặc trưng n-gram và pos tagging. Ngoài ra, nhóm tác giả nhận thấy kết quả bộ phân loại tăng lên khi kích thước tập dữ liệu lớn hơn.\\

Trong cùng lĩnh vực với luận án này, nhiều nghiên cứu phân tích cảm xúc trong ngữ cảnh y khoa cũng sử dụng phương pháp dựa trên học máy. Cảm xúc trong lĩnh vực này thường được hiểu khác nhau tùy vào nguồn dữ liệu phân tích.
Nghiên cứu \cite{xia09improving} giới thiệu phương pháp phân loại ý kiến bệnh nhân qua nhiều bước. Bước đầu tiên xây dựng bộ phân loại chủ đề dựa trên dữ liệu đã được gán nhãn chủ đề. Bước thứ 2 phân loại tính phân cực của các ý kiến dựa trên dữ liệu đã gán nhãn tính phân cực. Nhóm tác giả sử dụng giải thuật MNB, đạt kết quả F-measure là 0.67.
Nghiên cứu của nhóm tác giả Niu, Yun \cite{niu2005analysis} phân tích cảm xúc trên câu sử dụng giải thuật SVM với các câu thuộc lĩnh vực y khoa (\term{medical text}). Cảm xúc trong nghiên cứu này được hiểu như kết quả  được thể hiện trong câu. Mỗi câu được phân loại vào 1 trong 4 nhóm: tích cực, tiêu cực, trung tính hoặc không thể hiện kết quả. Kết quả tốt nhất nghiên cứu đạt được là 79.42\%. Cũng như đa số các nghiên cứu trong lĩnh vực này, ngoài các đặc trưng thường dùng (uni-gram, bi-gram, pos tagging), nhóm tác giả đề xuất đặc trưng thay đổi tình trạng (\term{Change phrase}) cùng với việc bổ sung kiến thức về lĩnh vực y khoa bằng việc sử dụng các khái niệm y học trong UMLS. 
\subsection{Phương pháp dựa trên từ vựng}
Phương pháp phân tích cảm xúc dựa trên từ vựng phụ thuộc lớn vào các nguồn từ vựng cảm xúc. Nguồn từ vựng cảm xúc, thường được hiểu như các từ điển, là các tập hợp các từ ngữ thể hiện cảm xúc, và đánh giá tính phân cực của 1 từ bằng 1 số thực. Các từ điển này có thể được xây dựng thủ công hoặc bán thủ công. Lợi thế của phương pháp này là không cần huấn luyện, từ đó không cần dữ liệu đã được đánh nhãn. Phương pháp này thường được sử dụng cho việc phân tích cảm xúc trên các loại văn bản thông thường: các bài viết trên blog, các bình luận về phim, sản phẩm, hoặc trên các diễn đàn. TODO hạn chế \\

Nghiên cứu \cite{ohana2009sentiment} sử dụng từ điển SentiWordNet để đánh giá tính phân cực của các bình luận phim. SentiWordNet là một từ điển được tạo tự động dựa trên cơ sở dữ liệu WordNet. Kết quả tốt nhất đạt được độ chính xác 69.35\%. Nhóm tác giả kết luận việc sử dụng từ điển SentiWordNet đạt hiệu quả tương đương với sử dụng từ điển được xây dựng bằng tay.\\

Một số nghiên cứu tự xây dựng bộ tử điển dựa trên các nguồn khác nhau. Nghiên cứu \cite{taboada2011lexicon} khẳng định việc xây dựng bộ điển trong nghiên cứu đó giúp tạo lập một nền tảng vững chắc cho hướng tiếp cận này. Nghiên cứu tự xây dựng bộ từ điển dựa trên bộ từ điển được xây dựng thủ công General Inquirer và một số nguồn văn bản khác. Nhóm tác giả hiện thực hệ thống SO-CAL nhằm tính điểm cho tài liệu dựa trên một tập các quy tắc với bộ từ điển đã xây dựng. Kết quả nghiên cứu cho thấy bộ từ điển được xây dựng tốt hơn các từ điển được xây dựng thủ công hoặc tự động trước đây như: Google dictionary, Maryland dictionary, SentiWordNet, \ldots. Một kết quả quan trọng hơn được nêu ra, hệ thống được xây dựng SO-CAL đạt hiệu suất tốt trên các loại bình luận khác nhau. Điều này có ý nghĩa quan trọng khi phương pháp phân tích cảm xúc trên văn bản thường bị phụ thuộc vào lĩnh vực, đặc biệt đối với phương pháp học máy \cite{Giachanou2016}\\

Trong cùng lĩnh vực với luận án này, nghiên cứu \cite{na2012sentiment} phân tích cảm xúc các bình luận về thuốc. Nghiên cứu phân tích trên mức mệnh đề. Nhóm tác giả tự xây dựng 2 bộ từ điển: từ điển cho các từ vựng thông thường và từ điển các từ chuyên ngành. Trong quá trình xây dựng, nhóm tác giả sử dụng các từ điển có sẳ:  Subjectivity Lexicon, SentiWordNet và từ điển các từ ngữ thông dụng được tập hợp từ dự án mã nguồn mở 12dict \footnote{http://wordlist.sourceforge.net/}. Sau đó, sử dụng công cụ xử lý ngôn ngữ NLP Stanford để tạo các quan hệ, xây dựng tập các luật để tính điểm. Kết quả tốt nhất đạt độ chính xác 0.78, tốt hơn thử nghiệm sử dụng phương pháp học máy mà nhóm tác giả thực hiện.
\subsection{Phương pháp kết hợp cả học máy và từ vựng}
Phương pháp dựa trên học máy gặp hạn chế về việc phụ thuộc vào kích thước tập huấn luyện. Tập huấn luyện, là tập dữ liệu đã được đánh nhãn, phải đủ lớn. Nhưng dữ liệu đã được đánh nhãn không phổ biến, và các nghiên cứu thường phải tự bỏ thời gian và chi phí thực hiện việc này. Trong khi đó, phương pháp dựa trên từ vựng tuy không cần bộ dữ liệu huấn luyện nhưng gặp hạn chế về mặt không phụ thuộc ngữ cảnh. Mỗi từ luôn thể hiện một tính phân cực như nhau trong mọi câu văn, điều thường không đúng với thực tế. Một cách để vượt qua các hạn chế này là kết hợp 2 phương pháp trên.\\

Nghiên cứu \cite{gonccalves2013comparing} tiến hành so sánh các phương pháp thuộc 2 nhóm phương pháp trên, sau đó đề xuất sự kết hợp. Kết quả tốt nhất đạt F-measure là 0.73. 

Trong cùng lĩnh vực với luận án này, nghiên cứu \cite{ali2013can} thực hiện phân tích cảm xúc sử dụng nguồn dữ liệu từ diễn đàn về sức khỏe. Cảm xúc trong các nguồn dữ liệu này thường đề cập đến ý kiến của người dùng về việc điều trị, về bác sĩ, hoặc cảm nhận của chính họ đối với sức khỏe của mình. Người dùng có thể phàn nàn rằng phương pháp điều trị gây hiệu ứng phụ, tuy nhiên kết quả điều trị vẫn tốt. Ali, Tanveer, et al \cite{ali2013can} sử dụng các phương pháp học máy với đặc trưng \textit{Bag-Of-Word} để phân loại cảm xúc về vấn đề ``mất thính giác'' với bộ dữ liệu được lấy từ ba diễn đàn y khoa\footnotemark. Sau đó, nghiên cứu kết hợp phương pháp dựa trên từ vựng bằng cách sử dụng từ điển \textit{Subjectivity Lexicon} giúp cải thiện kết quả hơn 4.2\%. \\

Liên quan nhất với phương pháp được sử dụng trong luận án này là nghiên cứu \cite{sarker2011outcome}. Nhóm tác giả sử dụng phương pháp học máy SVM, nhưng ngoài các đặc trưng thường dùng như n-gram, nhóm giới thiệu đặc trưng Hướng ngữ nghĩa (\term{Semantic Orientation}). Đặc trưng có bản chất là phương pháp dựa trên từ điển khi tính điểm số $SO$ cho tài liệu dựa trên trung bình cộng điểm số của các từ trong tài liệu đó. Điểm số các từ dựa theo từ điển General Inquirer. Kết luận của nghiên cứu khẳng định việc sử dụng đặc trưng Hướng ngữ nghĩa (\term{Semantic Orientation}) giúp tăng độ chính xác.
TODO  \cite{Zhang2011}