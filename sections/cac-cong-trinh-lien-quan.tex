\section{Các công trình liên quan}
Bài toán Phân tích cảm xúc (hay Khai phá ý kiến) đã được đặt ra từ trước thế kỷ XXI và nhanh chóng trở thành chủ đề hấp dẫn trong lĩnh vực Xử lý ngôn ngữ tự nhiên bởi tính ứng dụng cao. Chủ đề Phân tích cảm xúc được phân nhỏ thành nhiều bài toán con như: phân loại tính phân cực (tích cực hay tiêu cực), phân loại ý kiến chủ quan với ý kiến khách quan, phân tích biểu cảm (vui, buồn, giận, \ldots), \ldots\\

Về nguồn dữ liệu, bài toán Phân tích cảm xúc đã được nghiên cứu trên nhiều lĩnh vực khác nhau như bình luận phim, bình luận trên các trang blog, các bình luận về sản phẩm, bình luận trên các diễn đàn về sức khỏe, các \term{Tweet} trên mạng xã hội Twitter, các văn bản y khoa, \ldots Trong luận án này, chúng tôi chỉ quan tâm đến nguồn dữ liệu là các văn bản y khoa.\\

Những phương pháp được đề xuất để giải quyết bài toán có thể chia thành 3 nhóm\cite{Silva2015}: phương pháp dựa trên học máy, phương pháp dựa trên từ vựng và phương pháp sử dụng kết hợp cả hai (Hình \ref{fig:cac-pp-phan-tich}). Sau đây chúng tôi sẽ trình bày bối cảnh đề tài Phân tích cảm xúc dựa trên từng nhóm phương pháp.
%Các phương pháp được sử dụng cũng tương tự, tuy nhiên với các nguồn dữ liệu khác nhau, bài toán con này có những cách định nghĩa khác nhau về cảm xúc, từ đó dẫn đến việc sử dụng các phương pháp và kết quả cũng khác nhau.
\begin{figure}
\centering
\includegraphics[scale=0.25]{../hinh/cac_pp_phan_tich.png}
\caption{Các phương pháp phân tích cảm xúc trong ngữ cảnh chung}
\label{fig:cac-pp-phan-tich}
\end{figure}
\subsection{Phương pháp dựa trên học máy}
Nhiều phương pháp học máy có thể được áp dụng để giải quyết bài toán như \term{Naïve Bayes}, SVM (\term{Support Vector Machine}), \term{Maximun Entropy},\ldots trong đó nổi bật nhất là hai thuật toán học máy có giám sát \term{Naïve Bayes} và SVM. \cite{pang2002thumbs} đã thử nghiệm cả 3 giải thuật \term{Naïve Bayes}, SVM và \term{Maximun Entropy}, sử dụng các đặc trưng uni-gram, bi-gram và gán nhãn từ loại (\term{Part-of-speech Tagging}) để giải quyết bài toán phân loại nhị phân: phân loại các bình luận, nhận xét về phim có tính tính cực hay tiêu cực. Nghiên cứu đạt kết quả tốt nhất là 81.6\% với giải thuật SVM được sử dụng. Khác với \cite{pang2002thumbs}, Pak và Paroubek đã giải quyết bài toán phân loại đa lớp: phân tích xem các \term{tweet} trên Twitter thuộc loại tích cực, tiêu cực hay trung tính. Nghiên cứu đã thí nghiệm các bộ phân loại SVM, MNB (\term{Multinomial Naïve Bayes}), CRF (\term{Conditional  Random Field}) sử dụng đặc trưng n-gram (gồm uni-gram, bi-gram), và vị trí các n-gram. Kết quả tốt nhất nhóm tác giả đạt được khi kết hợp MNB với đặc trưng n-gram và gán nhãn từ loại. Ngoài ra, nhóm tác giả nhận thấy kết quả bộ phân loại tăng lên khi kích thước tập dữ liệu lớn hơn.\\

Trong lĩnh vực y khoa, nhiều nghiên cứu phân tích cảm xúc cũng sử dụng phương pháp dựa trên học máy. Cảm xúc trong lĩnh vực y khoa thường được hiểu khác nhau tùy vào nguồn dữ liệu phân tích.
Nghiên cứu \cite{xia09improving} giới thiệu phương pháp phân loại ý kiến bệnh nhân qua nhiều bước. Bước đầu tiên xây dựng bộ phân loại chủ đề dựa trên dữ liệu đã được gán nhãn chủ đề. Bước thứ 2 phân loại tính phân cực của các ý kiến dựa trên dữ liệu đã gán nhãn tính phân cực. Nhóm tác giả sử dụng giải thuật MNB, đạt kết quả $F-measure$ là 0.67.
Nghiên cứu của nhóm tác giả Niu, Yun \cite{niu2005analysis} phân tích cảm xúc trên câu sử dụng giải thuật SVM với các câu thuộc lĩnh vực y khoa (\term{medical text}). Cảm xúc trong nghiên cứu này được hiểu như kết quả  được thể hiện trong câu. Mỗi câu được phân loại vào 1 trong 4 nhóm: tích cực, tiêu cực, trung tính hoặc không thể hiện kết quả. Kết quả tốt nhất nghiên cứu đạt được là 79.42\%. Cũng như đa số các nghiên cứu trong lĩnh vực này, ngoài các đặc trưng thường dùng (uni-gram, bi-gram, gán nhãn từ loại), nhóm tác giả đề xuất sử dụng đặc trưng Chuyển đổi trạng thái (\term{Change phrase}) cùng với việc bổ sung kiến thức về lĩnh vực y khoa bằng cách sử dụng các khái niệm y học trong hệ thống UMLS (\term{Unified Medical Language System}). 
\subsection{Phương pháp dựa trên từ vựng}
Phương pháp phân tích cảm xúc dựa trên từ vựng phụ thuộc vào các nguồn từ vựng cảm xúc. Nguồn từ vựng cảm xúc, thường được hiểu như một bộ từ điển, là tập hợp các từ ngữ thể hiện cảm xúc với mỗi từ được đánh giá tính phân cực bằng một số thực. Các từ điển này có thể được xây dựng thủ công hoặc bán thủ công. Lợi thế của phương pháp này là không cần huấn luyện, từ đó không cần dữ liệu đã được đánh nhãn. Phương pháp này thường được sử dụng cho việc phân tích cảm xúc trên các loại văn bản thông thường: các bài viết trên blog, các bình luận về phim, sản phẩm, hoặc trên các diễn đàn. \\

Nghiên cứu \cite{ohana2009sentiment} sử dụng từ điển SentiWordNet để đánh giá tính phân cực của các bình luận phim. SentiWordNet là một từ điển được tạo tự động dựa trên cơ sở dữ liệu WordNet. Kết quả tốt nhất đạt được độ chính xác 69.35\%. Nhóm tác giả kết luận việc sử dụng từ điển SentiWordNet đạt hiệu quả tương đương với sử dụng từ điển được xây dựng bằng tay.\\

Một số nghiên cứu khác tự xây dựng bộ từ điển dựa trên các nguồn khác nhau. Nghiên cứu \cite{taboada2011lexicon} khẳng định việc xây dựng bộ từ điển giúp tạo lập một nền tảng vững chắc cho hướng tiếp cận này. Nghiên cứu tự xây dựng bộ từ điển dựa trên bộ từ điển được xây dựng thủ công \term{General Inquirer} và một số nguồn văn bản khác. Nhóm tác giả hiện thực hệ thống SO-CAL nhằm tính điểm cho tài liệu dựa trên một tập các quy tắc với bộ từ điển đã xây dựng. Kết quả nghiên cứu cho thấy bộ từ điển được xây dựng tốt hơn các từ điển được xây dựng thủ công hoặc tự động trước đây như: từ điển Google, từ điển Maryland, SentiWordNet, \ldots. Hơn nữa, kết quả nghiên cứu cho thấy hệ thống SO-CAL đạt hiệu suất tốt trên các bình luận thuộc nhiều lĩnh vực khác nhau. Điều này có ý nghĩa quan trọng khi phương pháp phân tích cảm xúc trên văn bản thường bị phụ thuộc vào lĩnh vực, đặc biệt đối với phương pháp học máy \cite{Giachanou2016}.\\

Trong lĩnh vực y khoa, nghiên cứu \cite{na2012sentiment} phân tích cảm xúc các bình luận về thuốc trên mức mệnh đề câu. Nhóm tác giả tự xây dựng 2 bộ từ điển: từ điển cho các từ vựng thông thường và từ điển các từ chuyên ngành. Trong quá trình xây dựng, nhóm tác giả đã sử dụng các từ điển có sẵn như \term{Subjectivity Lexicon}, SentiWordNet và từ điển các từ ngữ thông dụng được tập hợp từ dự án mã nguồn mở 12dict\footnote{http://wordlist.sourceforge.net/}. Sau đó, nhóm tác giả sử dụng công cụ xử lý ngôn ngữ NLP Stanford nhằm tạo các quan hệ và xây dựng tập các luật để tính điểm. Kết quả tốt nhất đạt độ chính xác $78\%$, tốt hơn thí nghiệm dùng phương pháp học máy mà nhóm tác giả đã thực hiện.
\subsection{Phương pháp kết hợp học máy và từ vựng}
Một trong những hạn chế của phương pháp dựa trên học máy là việc phụ thuộc vào kích thước tập huấn luyện - là tập dữ liệu đã được đánh nhãn và phải đủ lớn. Nhưng dữ liệu đã được đánh nhãn thường không phổ biến, đặc biệt trong lĩnh vực y khoa, và các nhóm nghiên cứu đa số phải tự bỏ thời gian và chi phí để đánh nhãn dữ liệu. Trong khi đó, phương pháp dựa trên từ vựng tuy không cần bộ dữ liệu huấn luyện nhưng gặp hạn chế vì không có tính chuyên sâu trên từng lĩnh vực cụ thể. Với phương pháp này, mỗi từ luôn thể hiện một tính phân cực như nhau trong mọi tình huống, điều này thường không đúng với thực tế. Một cách để vượt qua các hạn chế này là kết hợp cả 2 phương pháp trên.\\

Nghiên cứu \cite{gonccalves2013comparing} tiến hành so sánh các phương pháp thuộc 2 nhóm phương pháp trên, sau đó đề xuất sự kết hợp. Kết quả tốt nhất đạt $F-measure$ là 0.73. Nghiên cứu \cite{Zhang2011} giải quyết bài toán phân loại nhị phân. Ban đầu, nhóm tác giả sử dụng phương pháp dựa trên từ vựng để phân loại cảm xúc ở mức thực thể (\term{entity level}). Sau đó, một bộ phân loại sử dụng SVM được huấn luyện bằng chính tập dữ liệu được phân loại ở bước đầu. Kết quả đạt được độ chính xác $F$ bằng $0.749$.\\

Trong lĩnh vực y khoa, nghiên cứu \cite{ali2013can} thực hiện phân tích cảm xúc trên nguồn dữ liệu từ diễn đàn về sức khỏe. Cảm xúc trong các nguồn dữ liệu này thường đề cập đến ý kiến của người dùng về việc điều trị, về bác sĩ, hoặc cảm nhận của chính họ đối với sức khỏe của mình. Người dùng có thể phàn nàn rằng phương pháp điều trị gây hiệu ứng phụ, tuy nhiên kết quả điều trị vẫn tốt. Ali, Tanveer, et al \cite{ali2013can} sử dụng các phương pháp học máy với đặc trưng \textit{Bag-Of-Word} để phân loại cảm xúc về vấn đề ``mất thính giác'' với bộ dữ liệu được lấy từ ba diễn đàn y khoa\footnotemark. Sau đó, nghiên cứu kết hợp phương pháp dựa trên từ vựng bằng cách sử dụng từ điển \textit{Subjectivity Lexicon} giúp cải thiện kết quả hơn 4.2\%.\footnotetext{http://www.medhelp.org, http://www.alldeaf.com, http://www.hearingaidforums.com} \\

Liên quan gần nhất với phương pháp được sử dụng trong luận án của chúng tôi là nghiên cứu \cite{sarker2011outcome}. Nhóm tác giả sử dụng phương pháp học máy SVM, nhưng ngoài các đặc trưng thường dùng như n-gram, nhóm đề xuất sử dụng đặc trưng Hướng ngữ nghĩa (\term{Semantic Orientation} - SO). Bản chất của đặc trưng là phương pháp dựa trên từ điển: điểm số $SO$ cho tài liệu bằng trung bình cộng điểm số của các từ trong tài liệu đó. Điểm số các từ được tra cứu trong từ điển \term{General Inquirer}. Kết luận của nghiên cứu khẳng định việc sử dụng đặc trưng Hướng ngữ nghĩa giúp tăng độ chính xác.