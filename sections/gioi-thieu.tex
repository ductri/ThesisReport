\section{Tổng quan}
\subsection{Giới thiệu đề tài}
Trong lĩnh vực nghiên cứu khoa học nói chung và y học nói riêng, quá trình học hỏi, trau dồi kiến thức nền tảng và kiểm nghiệm lại nguồn tri thức đã có là sự khởi đầu cần thiết để tìm tòi, sáng tạo nên những tri thức mới đóng góp cho cộng đồng. Ngày nay, với sự phát triển mạnh mẽ của ngành Công nghệ thông tin, nguồn dữ liệu y học đang dần được số hóa và lưu trữ ở những kho dữ liệu trên khắp thế giới, điển hình là PMC\footnote{\term{PubMed Central} - https://www.ncbi.nlm.nih.gov/pmc/} và MEDLINE\footnote{https://www.nlm.nih.gov/pubs/factsheets/medline.html} - hai nguồn dữ liệu lớn thuộc Thư viện Y khoa Quốc gia Hoa Kỳ. Trong khi PMC hiện đã có hơn 4 triệu bài báo được cập nhật, thì kho dữ liệu MEDLINE đã tổng hợp được hơn 23 triệu tham khảo y học từ khắp các nguồn trên thế giới. Việc số hóa một lượng lớn dữ liệu y học góp phần tạo cơ hội tốt hơn, bình đẳng hơn cho mọi người tiếp cận với nguồn tri thức y học, nhưng đồng thời cũng đặt ra những thách thức mới trong việc tìm kiếm, chắt lọc, khai thác kho tàng tri thức quý báu này.\\

Trong hơn 10 năm trở lại đây, cuộc cách mạng mang tên ``Y học thực chứng'' (\term{Evidence-based medicine} - EBM) đã xảy ra trong giới y khoa làm thay đổi thói quen điều trị dựa trên cảm tính hay kinh nghiệm cá nhân của bác sĩ, thay vào đó là dựa vào các dữ kiện đáng tin cậy, đã qua kiểm tra một cách khoa học và có hệ thống \cite{Nguyen2004}. Rào cản lớn nhất của phương pháp Y học thực chứng hiện nay là tình trạng quá tải thông tin y khoa, bởi nó đòi hỏi người điều trị phải xem xét các tài liệu liên quan trước khi đưa ra quyết định lâm sàng. Khi tra cứu các báo cáo y khoa, người đọc chủ yếu quan tâm đến kết quả báo cáo và tính phân cực của cảm xúc trong kết quả này. Cực của cảm xúc (\term{sentiment polarity}) trong báo cáo y khoa có thể là \tichcuc (ví dụ như nghiên cứu cho thấy rằng thuốc \textit{X} mang lại hiệu quả tốt cho bệnh nhân bị bệnh \textit{Y}), \tieucuc (ví dụ như nghiên cứu cho thấy rằng thuốc \textit{X} không nên áp dụng cho bệnh nhân bị bệnh \textit{Y}), hoặc \trungtinh (ví dụ như nghiên cứu chỉ ra rằng thuốc \textit{X} có hiệu quả nhưng đi kèm với nhiều tác dụng phụ không mong muốn, hoặc nghiên cứu không đưa ra kết luận nào).\\

Việc tự tổng hợp và đánh giá kết quả của một lượng lớn báo cáo y khoa tiêu tốn không ít thời gian và kém hiệu quả tại thời điểm ra quyết định lâm sàng. Các nghiên cứu gần đây trong \cite{medhat2014sentiment} đã ghi nhận tầm quan trọng của việc phát triển hệ thống tự động phân tích tính phân cực cảm xúc bởi những lý do sau \cite{niu2005analysis}: 
\begin{itemize}
\item Đầu tiên, tính phân cực của cảm xúc trong báo cáo y khoa giúp trả lời câu hỏi về lợi ích hay tác hại của một can thiệp y tế (có thể là một phương pháp điều trị hay một loại thuốc,...).
\item Thứ hai, những trường hợp nghiên cứu không đưa ra kết quả nào giúp lọc bỏ những thông tin không cần thiết khi có câu hỏi về kết quả của một can thiệp y tế.
\item Thứ ba, kết quả \tieucuc mô tả tác dụng phụ có thể rất quan trọng cho một quyết định lâm sàng ngay cả khi không được hỏi tới.
\item Cuối cùng, từ một loạt các kết quả \tichcuc hay \tieucuc của một can thiệp y tế vào một bệnh lý cụ thể, người điều trị có thể có cái nhìn tổng quát hơn về độ phù hợp khi sử dụng can thiệp đó cho quyết định lâm sàng.
\end{itemize}

Bên cạnh đó, hệ thống phân tích tính phân cực cảm xúc ngoài mục đích hỗ trợ cho việc ra quyết định lâm sàng còn tạo nền tảng cho các hệ thống khác rút trích, tổng hợp nhằm phát hiện ra những tri thức mới tiềm ẩn trong kho dữ liệu y học to lớn của nhân loại. Đó là lý do chúng tôi quyết định chọn đề tài ``Phân tích cảm xúc trong văn bản y khoa'' làm đề tài Luận văn tốt nghiệp.

\subsection{Mục tiêu và phạm vi đề tài}


\subsection{Cấu trúc luận văn}
