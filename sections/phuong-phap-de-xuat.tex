\section{Phương pháp đề xuất}
\subsection{Mô tả bài toán}
\subsection{Kiến trúc tổng quan}
\subsection{Các đặc trưng cơ bản}
\subsubsection*{N-gram}
Theo kết luận của nghiên cứu \cite{chandrakala2012opinion}, n-gram là đặc trưng được sử dụng phổ biến trong bài toán phân tích cảm xúc nói chung. Nhiều nghiên cứu về phân tích cảm xúc trong lĩnh vực y khoa cũng sử dụng đặc trưng này \cite{pang2002thumbs}, \cite{niu2005analysis}, \cite{sarker2011outcome}, \cite{niu2006using}, \cite{pestian2012sentie}, \cite{xia09improving} \\

%Định nghĩa
N-gram là một chuỗi gồm n phần tử liên tiếp nhau. Các phần tử này có thể chữ cái, âm tiết hoặc đoạn văn\ldots Trong nghiên cứu này, các phần tử là các từ đơn trong câu. Đơn vị từ được định nghĩa là chuỗi các chữ cái liên tiếp nhau không chứa ký tự khoảng trắng, các từ phân biệt nhau bởi ký tự khoảng trắng. Đặc trưng n-gram đóng góp lớn trong kết quả của các phân tích, vì vậy đặc trưng này thường được dùng như baseline. Báo cáo  của \cite{pang2002thumbs} phân tích cảm xúc trên các bình luận về phim, đạt độ chính xác 82.9\% với chỉ một đặc trưng n-gram. Đây cũng là kết quả tốt nhất của nghiên cứu này. Nghiên cứu của \cite{niu2005analysis} đạt độ chính xác 77.87\% khi chỉ sử dụng n-gram như \term{baseline}, kết quả tốt nhất tăng 20.58\% so với \term{baseline}. \\

Ví dụ câu: ``Standard practice in pupillary monitoring yields inaccurate data''
\begin{itemize}
\item[•]Với $n=1$, n-gram được gọi là uni-gram. Câu trên sẽ được chuyển thành các n-gram: Standard, practice, in, pupillary, monitoring, yields, inaccurate, data.
\item[•]Với $n=2$, n-gram được gọi là bi-gram. Câu trên sẽ được chuyển thành các n-gram: Standard practice, practice in, in pupillary, pupillary monitoring, monitoring yields, yields inaccurate, inaccurate data.
\item[•]Với $n=3$, n-gram được gọi là tri-gram. Câu trên sẽ được chuyển thành các n-gram: Standard practice in, practice in pupillary, in pupillary monitoring, pupillary monitoring yields, monitoring yields inaccurate, yields inaccurate data.
\item[•]Với $n>3$, tần suất xuất hiện các n-gram thấp, dễ làm mô hình học máy bị học quá khớp (overfitting)
\end{itemize}

Việc phối hợp các n-gram là tùy chọn đối với mỗi nghiên cứu, và các kết quả cũng không hoàn toàn đồng nhất. Báo cáo \cite{niu2005analysis} kết luận khi sử dụng bi-gram kết hợp với uni-gram giúp tăng độ chính xác thêm 3.01\%, điều này phù hợp với báo cáo \cite{sarker2011outcome}. Báo cáo \cite{sarker2011outcome} kết luận rằng việc dùng cả uni-gram, bi-gram và tri-gram giúp cải thiện kết quả rõ rệt. Trong khi đó \cite{pang2002thumbs} đạt kết quả cao nhất chỉ với uni-gram. Kết luận của \cite{pang2002thumbs} cho thấy việc sử dụng thêm đặc trưng bi-gram không tác động nhiều đến kết quả. Báo cáo của \cite{smith2012cross} phân tích cảm xúc trong lĩnh vực y khoa lâm sàng, khẳng định đặc trưng uni-gram cho kết quả tốt hơn bi-gram. Tuy nhiên, nghiên cứu không thử nghiệm kết hợp 2 đặc trưng này. Trong báo cáo này, chúng tôi thử nghiệm các cách kết hợp khác nhau của n-gram để tìm ra kết quả tốt nhất.\\

%\subsubsection*{Mở rộng}
TODO Việc áp dụng các kiến thức liên quan đến lĩnh vực đang xem xét giúp tăng độ chính xác khi phân loại. Tác giả Kerstin Denecke trong nghiên cứu \cite{denecke2015sentiment} khẳng định rằng áp dụng kiến thức trong lĩnh vực y khoa là cần thiết để cải thiện hiệu quả phân loại cảm xúc. Trong nghiên cứu này, ý nghĩa cụ thể của các thuật ngữ y khoa không có tác dụng phân loại cảm xúc, chỉ thông tin mô tả của các thuật ngữ này có ý nghĩa.\\

\example{}{\myquote{``Elevated troponin level after acute stroke is common and is associated with ECG changes suggestive of myocardial ischemia and increased risk of death''}\\
Từ \myquote{stroke} xét trong ngữ cảnh y học nghĩa là đột quỵ. Nhưng đối với việc phân loại cảm xúc, chúng tôi chỉ quan tâm tới ý nghĩa khái quát của từ này: \myquote{stroke} mô tả một loại bệnh. Tương tự các từ \myquote{diarrhoea, abdominal pain, nausea} chỉ cần được hiểu như vấn đề về bụng mà không cần hiểu cụ thể như thế nào.}

Như vậy, các thuật ngữ y khoa thuộc cùng 1 kiểu (triệu chứng, loại bệnh, tên thuốc, ...) đều được xem là một. Từ đó, giảm thiểu khả năng bộ phân loại bị nhiễu hoặc bị học lệch (\term{overfitting}).\\

Một trong những công cụ đã được xây dựng hoàn chỉnh và sử dụng phổ biến trong các bài toán liên quan đến y khoa trên dữ liệu tiếng Anh là UMLS. Đây là một hệ thống tích hợp các thuật ngữ y khoa cùng các mã chuẩn hóa nhằm tạo tiền đề cho việc xây dựng và phát triển các hệ thống thông tin y khoa cũng như các dịch vụ chăm sóc y tế khác. Để hiện thực nhiệm vụ trên, chúng tôi sử dụng công cụ MetaMap kết hợp hệ thống UMLS. UMLS phân các thuật ngữ y học ra làm 136 kiểu \footnote{Chi tiết các kiểu tham khảo tại \texttt{https://metamap.nlm.nih.gov/Docs/SemanticTypes\_2013AA.txt}}. MetaMap là một công cụ cho phép tra cứu tên kiểu của 1 thuật ngữ y học bất kỳ (Hình \ref{fig:metamap}).\\

\begin{figure}[!]
\centering
\includegraphics[scale=0.32]{metamap.png}
\caption{MetaMap sử dụng nguồn tài nguyên UMLS, giúp tra cứu tên kiểu của 1 thuật ngữ y học}
\label{fig:metamap}
\end{figure}
%Một số ví dụ:\\
\example{1}{
\myquote{Elevated troponin level after acute \underline{stroke} is common and is associated with ECG changes suggestive of myocardial \underline{ischemia} and increased risk of death}\\
$\xrightarrow{MetaMap}$ \\
\myquote{Elevated troponin level after acute \underline{DSYN} is common and is associated with ECG changes suggestive of myocardial \underline{DSYN} and increased risk of death}\\
Từ \underline{\myquote{stroke}} và \underline{\myquote{ischemia}} đều thuộc kiểu loại bệnh hoặc triệu chứng, nên được thay bằng nhãn DSYN (Disease or Syndrome)
}
\example{2}{
\myquote{
In addition, regardless of psychiatric status, \underline{ADHD} placed children at relative risk for educational and vocational disadvantage}\\
$\xrightarrow{MetaMap}$ \\
\myquote{In addition, regardless of psychiatric status, \underline{MOBD} placed children at relative risk for educational and vocational disadvantage}\\
Từ \underline{\myquote{ADHD}} chỉ một dạng rối loạn về thần kinh, nên được thay bằng nhãn MOBD (Mental or Behavioral Dysfunction)\\
}
\subsubsection*{Change Phrase}
Đặc trưng change phrase được Niu, Yun et al. định nghĩa trong một nghiên cứu phân tích cảm xúc trên câu \cite{niu2005analysis}. Sau đó được nhóm tác giả Sarker, Abeed, et al. sử dụng lại. Bài toán mà Sarker, Abeed, et al giải quyết cũng tương tự nhưng thay vì phân tích trên câu, nhóm tác giả phân tích trên đoạn. Ngoài việc sử dụng lại, Saker, Abeed,et al. có một số thay đổi và mở rộng đặc trưng này.\\

\textit{Change phrase} là những cụm từ mang ý nghĩa làm thay đổi tình trạng, trạng thái: làm tốt hơn hoặc làm tệ hơn. Tính phân cực trong một câu thường biểu thị qua sự thay đổi \cite{niu2005analysis}, và hay xuất hiện ở những câu so sánh. \\

\example{}{Câu \myquote{Atypical antipsychotic use is associated with an increased risk for death compared with nonuse among older adults with dementia} thể hiện một tình trạng tệ hơn: Sử dụng \myquote{Atypical antipsychotic} làm tăng nguy cơ chết so với không sử dụng \myquote{Atypical antipsychoti'}.}
Chúng tôi sử dụng 4 nhóm để mô tả \textit{Change phrase}:
\begin{itemize}
\item[•]Nhóm thể hiện sự thay đổi tình trạng, gồm 2 nhóm:\\
LESS: Có ý nghĩa làm giảm bớt, hạ bớt. Một số từ như: ``reduce'', ``decline'', ``fall'', ``less'', ``little'',\ldots \\
MORE: Có ý nghĩa ngược lại, làm tăng thêm (hoặc duy trì). Một số từ như: ``enhance'', ``higher'', ``exceed'', ``increase'', ``improve'',\ldots
\item[•]Nhóm xác định tính phân cực, gồm 2 nhóm:\\
GOOD: Mang ý nghĩa tích cực. Một số ví dụ như: ``benefit'', ``improvement'', ``advantage'', ``accuracy'', ``great'', \ldots\\
BAD: Mang ý nghĩa tiêu cực. Một số ví dụ như: ''suffer``, ''adverse``, ''hazards``, ''risk``, ''death``,\ldots
\end{itemize}
Danh sách các từ cho mỗi nhóm trên được chúng tôi tập hợp thủ công. Kết hợp 4 nhóm trên, ta có 4 đặc trưng giúp mô tả những thay đổi tích cực hoặc tiêu cực như \refformat{Bảng~\ref{tab:changphrase}}. \\

\example{1}{
\myquote{Atypical antipsychotic use is associated with an increased risk for death compared with nonuse among older adults with dementia}\\
Từ \myquote{increased} sẽ được gán nhãn MORE, \myquote{risk} được gán nhãn BAD, sau đó việc phân tích sẽ xác định được đối tượng của từ \myquote{increased} là \myquote{risk}. Từ đó, câu trên được nhận dạng thuộc mẫu MORE-BAD, suy ra nó có xu hướng biểu thị tính phân cực \tieucuc.}

\begin{table}[H]
\centering
\caption{Các đặc trưng \textit{Change phrase}}
\label{tab:changphrase}
\begin{tabular}{ | P{0.25\textwidth} | P{0.25\textwidth}| P{0.25\textwidth} | }
\hline
\textbf{Nhóm làm thay đổi tình trạng} & \textbf{Nhóm xác định đối tượng} & \textbf{Phân loại tính phân cực} \\
\hline
LESS & GOOD & \tieucuc \\
\hline
LESS & BAD & \tichcuc \\
\hline
MORE & GOOD	& \tichcuc \\
\hline
MORE & BAD & \tieucuc \\
\hline
\end{tabular}
\end{table}
\subsubsection*{Sự phủ định}
\subsection{Đặc trưng mở rộng SO-CAL}
Trong nghiên cứu này, chúng tôi sử dụng đặc trưng SO-CAL dựa trên bài báo \cite{taboada2011lexicon}, tên của đặc trưng xuất phát từ tên hệ thống mà bài báo trên đã xây dựng. Theo ý nhóm tác giả \cite{taboada2011lexicon}, SO-CAL là chữ viết tắt của Semantic orientation Calculator. Đây là một phương pháp phân tích cảm xúc dựa trên từ vựng.\\

Phân tích cảm xúc dựa trên từ vựng là một phương pháp khác, tránh được bất lợi của phương pháp học máy là không cần qua quá trình huấn luyện. Tuy nhiên đa số các nghiên cứu này đều phân tích cảm xúc trên văn bản thông thường, không tập trung vào 1 lĩnh vực cụ thể nào \cite{taboada2011lexicon}\cite{Zhang2011}\cite{ohana2009sentiment}\cite{Giachanou2016}. \\

Phương pháp này ngầm định 2 giả thiết sau đã được thỏa mãn:
\begin{itemize}
\item[•] Bản thân mỗi từ có sẳn tính phân cực mà không bị phụ thuộc vào ngữ cảnh. Điều này có nghĩa là mỗi từ luôn chỉ có 1 xu hướng phân cực (tốt, xấu hoặc tích cực, tiêu cực) trong mọi câu mà từ đó xuất hiện.
\item[•] Tính phân cực của mỗi từ được đề cập ở trên có thể được biểu diễn bởi 1 số thực
\end{itemize}
Dựa trên 2 giả thiết trên, tính phân cực của một câu phụ thuộc vào số thực biểu diễn tính phân cực của các từ trong câu đó, và cũng được biểu diễn bởi 1 số thực. Sự phụ thuộc giữa tính phân cực của các từ và tính phân cực của cả câu là tùy thuộc vào các nghiên cứu, có thể mô hình hóa như công thức sau:
\begin{equation}
Polarity_{sentence}=f(Polarity_{words-in-sentence})
\end{equation}

